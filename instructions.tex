\documentclass[a4paper]{article}
\usepackage[numbers]{natbib}
\usepackage{listings}
\usepackage[hidelinks]{hyperref}
\usepackage[T1]{fontenc} %Stops attempting to use speech marks in code listings.
\lstset{
    columns=fullflexible,
    basicstyle=\ttfamily,
    frame=single,
    breaklines=true,
}
\hyphenation{Game-scope}

\title{Arch Linux Post-Install}
\author{Wartybix}
\date{\today}

\begin{document}

\maketitle

These are some notes I have written for myself, for the next time I install Arch Linux.

\section{Using Pacman}

Install new packages with:
\begin{lstlisting}[escapechar=@]
# pacman -Syu @\underline{packagename}@
\end{lstlisting}
Passing the \lstinline|-Syu| flag updates all packages on the system, and is necessary to prevent any breakage from incompatible packages.
Check the manual page (\lstinline|man pacman|) for more information on what each of these 
flags do.

Remove packages with:
\begin{lstlisting}[escapechar=@]
# pacman -Rs @\underline{packagename}@
\end{lstlisting}
This removes (\lstinline|-R|) the package, as well any of its dependencies (\lstinline|-s|) not needed by other packages.

\section{Bash Autocomplete}

On a fresh installation of Arch Linux, the terminal doesn't always auto-complete when pressing tab while inputting package names and such.
Install the \lstinline|bash-completion| package for this functionality (\lstinline|pacman -Syu bash-completion|).

\section{Installing Yay}

To make management of AUR packages easier, Yay should be installed.
On a fresh install of Arch Linux, Git may not be installed --- in which case install it.
\begin{lstlisting}
# pacman -Syu git
\end{lstlisting}
Clone Yay's repository located at \url{https://aur.archlinux.org/yay.git}:
\begin{lstlisting}
$ git clone https://aur.archlinux.org/yay.git
\end{lstlisting}
Navigate into the new directory:
\begin{lstlisting}
$ cd yay
\end{lstlisting}
Install Yay:
\begin{lstlisting}
$ makepkg -si
\end{lstlisting}
Yay is now installed.

\section{Update Notifier}

This step is only helpful if using the GNOME desktop environment.

Get informed of updates for your system without needing to manually check yourself in the terminal.
To enable this functionality, install the Arch Linux Updates Indicator (\lstinline|arch-update@RaphaelRochet|) GNOME extension.

Disable the `Always visible' option in its settings to only show the update icon when new updates are available.

\subsection{Implementing with Yay}

By default, the extension only works with packages using Pacman.
Extend this functionality to Yay to allow the notifications and upgrades of new AUR package updates.
The following commands are as per the project's wiki (\url{https://github.com/RaphaelRochet/arch-update/wiki}).

In `Advanced Settings' of the extension's settings, change `Command to check for package updates' to:
\begin{lstlisting}
/bin/sh -c "(/usr/bin/checkupdates; /usr/bin/yay -Qu --color never | sed 's/Get .*//') | sort -u -t' ' -k1,1"
\end{lstlisting}
And change `Command to update packages' to:
\begin{lstlisting}
kgx -e '/bin/sh -c "yay ; echo Done - Press enter to exit; read"'
\end{lstlisting}

\section{Setting up Pacman cache auto-clean}

Pacman's cache of previous package versions will build up over time.
To clear this cache on a weekly basis, enable and start \lstinline|paccache.timer| \cite{paccache-timer}.

\begin{lstlisting}
# systemctl enable paccache.timer
# systemctl start paccache.timer
\end{lstlisting}

\section{Setting up secure boot}

Thank you to `Walian' on YouTube for an easy-to-follow guide on enabling secure boot on Arch Linux (\url{https://www.youtube.com/watch?v=yU-SE7QX6WQ}).
I have written the following instructions from this video here in order to preserve them, and to allow the following of these instructions in a textual manner.

Reboot into your motherboard's BIOS, either by entering your motherboard's assigned `BIOS key' on boot, or by entering the following in a Linux terminal:
\begin{lstlisting}
$ systemctl reboot --firmware-setup
\end{lstlisting}

In the BIOS, disable secure boot (if not disabled already), and reset your secure boot keys.
Save changes and reboot.

Install the \lstinline|sbctl| package:
\begin{lstlisting}
# pacman -Syu sbctl
\end{lstlisting}

Create secure boot keys with:
\begin{lstlisting}
# sbctl create-keys
\end{lstlisting}

Enroll the keys with your system with:
\begin{lstlisting}
# sbctl enroll-keys -m
\end{lstlisting}

Sign the bootloader with:
\begin{lstlisting}
# sbctl sign -s -o /usr/lib/systemd/boot/efi/systemd-bootx64.efi.signed /usr/lib/systemd/boot/efi/systemd-bootx64.efi
\end{lstlisting}

Find the location of your kernel image with:
\begin{lstlisting}
# cat /etc/mkinitcpio.d/linux.preset 
\end{lstlisting}

The location of the kernel image to sign is listed as \lstinline|ALL_kver=|.
This location will most likely be \lstinline|/boot/vmlinuz-linux|.
If using a unified kernel image, the kernel image to sign is listed as \lstinline|default_uki=|, which will most likely be \lstinline|/efi/EFI/Linux/arch-linux.efi|.

Sign the kernel image:
\begin{lstlisting}
# sbctl sign -s /boot/vmlinuz-linux
\end{lstlisting}
If using a unified kernel image:
\begin{lstlisting}
# sbctl sign -s /efi/EFI/Linux/arch-linux.efi
\end{lstlisting}
If using an LTS kernel image:
\begin{lstlisting}
# sbctl sign -s /boot/vmlinuz-linux-lts
\end{lstlisting}

Reinstall the bootloader:
\begin{lstlisting}
# bootctl install
\end{lstlisting}

Ensure all files have been signed correctly with:
\begin{lstlisting}
# sbctl verify
\end{lstlisting}

If everything is signed and there are no issues, reboot back into the BIOS and enable secure boot.

Reboot back into Arch Linux, and enter the following in a terminal to check secure boot status:
\begin{lstlisting}
$ sbctl status
\end{lstlisting}

Setup mode should be \emph{disabled} and secure boot \emph{enabled}.

Now, secure boot should be enabled, and Pacman should re-sign the kernel images when installing new packages.

\section{Enabling Bluetooth}

Install the \lstinline|bluez| and \lstinline|bluez-utils| packages:
\begin{lstlisting}
# pacman -Syu bluez bluez-utils
\end{lstlisting}

Then enable and start the Bluetooth service:
\begin{lstlisting}
# systemctl enable bluetooth.service
# systemctl start bluetooth.service
\end{lstlisting}

\section{Nano linting}

Enable linting in Nano by creating a file at \lstinline|~/.nanorc|.
In this file, add the following:
\begin{lstlisting}
include /usr/share/nano/*.nanorc
\end{lstlisting}
This enables linting with all of Nano's supported file types.

\section{Gaming on Linux}

\subsection{Using Xbox game controllers}

The default \lstinline|xpad| drivers for Linux aren't too good --- for example, there are issues with the latency of controller haptics.
Instead, install \lstinline|linux-headers| package with Pacman.
Then install \lstinline|xpadneo| with Yay.

The \lstinline|xpadneo| driver allows the use of rumble motors in the Xbox controller's triggers --- this is a feature unused in most games even when running Windows.
The driver allows the triggers to rumble at the same time as other haptic motors in the controller when being pressed.
However, the default trigger rumble attenuation might be a bit too strong to the point of distraction.
This strength of the trigger rumble can be configured in real time by writing \lstinline|0,n|  where \lstinline|n| is a value from 0 (highest strength) to 100 (lowest strength / off) to a file at \lstinline|/sys/module/hid_xpadneo/parameters/rumble_attenuation|, \cite{xpadneo-troubleshooting}.
I recommend a value of 70 (i.e., 30\% rumble strength) for a subtle-yet-noticeable haptic effect on the triggers --- your preferences may differ however.

To make this trigger rumble attention setting apply at loading time, enter the following: \cite{xpadneo-troubleshooting}

\begin{lstlisting}
$ echo "options hid_xpadneo rumble_attenuation=0,70" | sudo tee /etc/modprobe.d/99-xpadneo-bluetooth.conf
\end{lstlisting}
Change the \lstinline|rumble_attenuation=0,70| value to your preferred value.

\subsection{Feral GameMode}

Install Feral GameMode for CPU optimizations when playing games on Linux.
Install the \lstinline|gamemode| and \lstinline|lib32-gamemode| packages \cite{arch-wiki-gamemode}:

\begin{lstlisting}
$ sudo pacman -Syu gamemode lib32-gamemode
\end{lstlisting}

Add yourself to the \lstinline|gamemode| user group to give Feral GameMode's user daemon the rights to change the CPU governor \cite{arch-wiki-gamemode}.
\begin{lstlisting}[escapechar=@]
# usermod -aG gamemode @\underline{exampleuser}@
\end{lstlisting}

\subsection{Gamescope}

Gamescope is a compositor that allows for the customization of how games are output to the display.
Gamescope is especially useful when using fractional scaling in Wayland, as, without it, games detect a ``native'' resolution far beyond the capabilities of the monitor due to the scaling.
In this case the user must either use the `supersampled' resolution --- forcing their GPU to work harder than necessary, potentially resulting in a lower framerate --- or lower the in-game resolution to that of their monitor's (if even listed), resulting in a blurry and often glitchy image on Wayland.

Gamescope alleviates this issue as the user can simply enter their resolution with Gamescope's \lstinline|-H| and \lstinline|-W| flags.
To use Gamescope, first install it with the package manager:
\begin{lstlisting}
# pacman -Syu gamescope
\end{lstlisting}

Gamescope can then be used as part of games' launch options in the user's library.
Here is my recommended launch options to open most games:
\begin{lstlisting}[escapechar=@]
gamemoderun gamescope -H @\underline{monitor-height}@ -W @\underline{monitor-width}@ -r @\underline{monitor-refresh-rate}@ -f -e --backend sdl --force-grab-cursor -g %command%
\end{lstlisting}
\lstinline|gamemoderun| starts the Feral GameMode daemon as introduced last section.
\lstinline|gamescope| starts a Gamescope session of whatever \lstinline|%command%| is (i.e., opening a particular game).
Between \lstinline|gamescope| and \lstinline|%command%| in the launch options are \lstinline|gamescope|'s arguments:
\begin{itemize}
    \item \lstinline|-H| takes the monitor's height in pixels.
    \item \lstinline|-W| takes the monitor's width in pixels.
    \item \lstinline|-r| takes the monitor's refresh rate in Hz.
    \item \lstinline|-f| opens a full-screen window.
    \begin{description}
        \item[Note:] Sometimes this flag doesn't work, and Libdecor decorations on your Gamescope window and elements of your desktop environment will still be visible regardless. Use Gamescope's super + F keyboard shortcut to enter full-screen when this happens.
    \end{description}
    \item \lstinline|-e| adds Steam overlay integration. Some Steam games break when using Gamescope without this \cite{arch-wiki-gamescope}.
    \item \lstinline|--backend sdl| forces Gamescope to use SDL2 instead of automatically picking the Wayland backend. The Wayland backend has issues with micro-stutter and Vsync at time of writing, and it took me a while to find the cause of these issues after moving to Arch Linux (and hence using a more up-to-date version of Gamescope). \lstinline|mangoapp| was reporting a steady, high framerate when using the Wayland backend, yet my eyes saw a very jittery image, particularly when using Vsync in game. After doubting my sanity for a few hours, I realized the issue was with the relatively new Wayland backend, and switching back to SDL2 fixed the issue and the image was smooth again. I assume there's some kind of weirdness going on between Gamescope's nested compositor and Mutter or something, which is why Gamescope's in-built \lstinline|mangoapp| reported a flawless smooth framerate, but what was output was a visually jittery mess. Anyway just use SDL2 for now. Maybe the Wayland backend will get better in a year or so.
    \item \lstinline|--force-grab-cursor| Allows the window inside Gamescope to grab the mouse cursor. Allows moving a camera with the mouse without any limitations as one would expect, without the mouse (invisibly) hitting the edge of the display causing the camera movement to be limited.
    \item \lstinline|-g| Grabs the keyboard. Useful as Gamescope won't recognise its keyboard shortcuts (like super + F to toggle fullscreen) without it.
    \begin{description}
        \item[Note:] Your desktop environment's shortcut for toggling full-screen does \emph{not} achieve the same functionality as Gamescope's own shortcut. Your desktop environment's shortcut will stretch the output of Gamescope, making the image slightly blurry/pixelated. However Gamescope's own shortcut keeps the image intact. This is why it is important to pass the \lstinline|-g| flag to inhibit desktop keyboard shortcuts.
    \end{description}
\end{itemize}

\section{Using FAT32 and NTFS storage on Linux}

Install the \lstinline|dosfstools| and \lstinline|ntfsprogs| packages for compatibility with FAT32 and NTFS systems respectively.

\bibliographystyle{IEEEtranN}
\bibliography{refs}

\end{document}